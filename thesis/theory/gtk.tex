\emph{GTK Broadway} jest młodym projektem powstałym w 2011 roku stworzonym przez Alexandra Larssona. Projekt nie posiada pełnej dokumentacji, dostępne są jedynie krótkie wpisy na blogu\cite{broadway1,broadway2}.

Broadway analogicznie jak opisywana praca pozwala na uruchomienie aplikacji \emph{GTK} w przeglądarce. Broadway odwzorowuje każde okno aplikacji na jeden element \emph{canvas}. Zawartość okna jest aktualizowana poprzez komendy pobierane za pomocą żądania \emph{XHR} (XMLHttpRequest) emulując w ten sposób HTTP pushing. Dodatkowo komendy są kompresowane przy pomocy algorytmu gzipem. Dane z wejścia przeglądarki są zbierane poprzez zdarzenia modelu \emph{DOM} i wysyłane do serwera z wykorzystaniem mechanizmu \emph{WebSocketu}. Do przeglądarki przesyłane są obrazy \emph{PNG} jako \emph{data-URI} zakodowane w \emph{base64}. Zawierają one widoczne regiony, które zostały zmodyfikowane lub nie były wcześniej wyświetlone. Jest to główna różnica między projektami -- w rozwiązaniu przedstawionym w pracy dane przesyłane są w całkowicie inny sposób, szczegółowo opisany w kolejnych rozdziałach.

Backend ten nie jest domyślnie włączony w żadnej dużej dystrybucji systemu Linux i jego instalacja nie jest prosta. Wymaga ręcznej kompilacji całego pakietu \emph{GTK} z odpowiednimi flagami. Jest to bardzo kłopotliwe, gdyż wymaga to instalacji wielu zależności.