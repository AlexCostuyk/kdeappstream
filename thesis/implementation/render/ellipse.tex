\begin{lstlisting}[language=C++,numbers=none]
virtual void QPaintEngine::drawEllipse( const QRectF & rect );
virtual void QPaintEngine::drawEllipse( const QRect & rect );
\end{lstlisting}
\begin{lstlisting}[language=JavaScript,numbers=none]
{
  "t":"ellipse",
  "x":0.0,       // Pozycja srodka X
  "y":0.0,       // Pozycja srodka Y
  "w":10.0,      // Srednica pozioma
  "h":10.0       // Srednica pionowa
}
\end{lstlisting}

Pełna implementacja za pomocą krzywych Beziera.

\begin{lstlisting}[language=JavaScript,numbers=none,caption=Implementacja rysowania elips za pomocą krzywych Beziera wykorzystana w projekcie]
// ctx - konteks canvas, x,y,w,h - jak w JSON
var kappa = .5522848;
var ox = (w / 2) * kappa,
  oy = (h / 2) * kappa,
  xe = x + w,
  ye = y + h,
  xm = x + w / 2,
  ym = y + h / 2;
ctx.beginPath();
ctx.moveTo(x, ym);
ctx.bezierCurveTo(x, ym - oy, xm - ox, y, xm, y);
ctx.bezierCurveTo(xm + ox, y, xe, ym - oy, xe, ym);
ctx.bezierCurveTo(xe, ym + oy, xm + ox, ye, xm, ye);
ctx.bezierCurveTo(xm - ox, ye, x, ym + oy, x, ym);
ctx.closePath();
\end{lstlisting}