\begin{lstlisting}[language=C++,numbers=none]
virtual void QPaintEngine::updateState(const QPaintEngineState& state)
\end{lstlisting}
\begin{lstlisting}[language=JavaScript,numbers=none]
{
  "t":"state",
  "data":
    {
      // Opis pedzla krawedzi
      "pen": { /* ... */ },
      
      // Opis pedzla wypelnienia 
      "brush": { /* ... */ },
      
      // Czcionka
      "font": "Opis czcionki w formacie CSS",

      // Offset pedzla
      "brushorigin":
        {
          "x":0.0,
          "y":0.0
        },

      // Macierz transformacji 3x3
      "transform":[[1,0,0],[0,1,0],[0,0,1]], 
      
      // Metoda kompozycji
      "composition":"source-over",
      
      // Opis pedzla tla
      "bbrush": { /* ... */ },
      
      // Przezroczystosc
      "opacity":0.5,
      
      // Obcinanie
      "clip": { /* ... */ }
    }
}
\end{lstlisting}

\subsubsection{Pędzele wypełnienia i tła}

\begin{itemize}
\item Brak wypełnienia
\begin{lstlisting}[language=JavaScript,numbers=none]	    "brush":
  {
    "style":0,
  }
\end{lstlisting}

Pełna implementacja za pomocą ustawienia koloru na RGB(0,0,0,0) -- całkowicie przeźroczysty.

\item Gradient liniowy
\begin{lstlisting}[language=JavaScript,numbers=none]	    "brush":			
{
  "style":15,
  "gradient":
    {  
      "type":0,
      "xs":0.0, // Punkt poczatkowy X
      "ys":0.0, // Punkt poczatkowy Y
      "xe":0.0, // Punkt koncowy X
      "ye":0.0, // Punkt koncowy Y
      "stops":  // Lista kolorow (pary [odleglosc, kolor])
                // Ogleglosc wzgledna z przedzialu (0;1)
        [
          [0,"#FFFFFF"],
          [1,"#000000"]
        ],
      "spread":0, // Wypelnienie poza obszarem gradientu
                  // 0: QGradient::PadSpread
                  // 1: QGradient::ReflectSpread
                  // 2: QGradient::RepeatSpread
      "mode":0    // Definiuje sposob interpretacji wspolrzednych
                  // 0: QGradient::LogicalMode
                  // 1: QGradient::StretchToDeviceMode
                  // 2: QGradient::ObjectBoundingMode
    }
  "transform":[[1,0,0],[0,1,0],[0,0,1]] // Opcjonalne
}
\end{lstlisting}

Pełna implementacja za pomocą \emph{createLinearGradient}.

\item Gradient kołowy
\begin{lstlisting}[language=JavaScript,numbers=none]	    "brush":			
{
  "style":16,
  "gradient":
    {  
      "type":1,
      "xc":0.0, // Punkt srodkowy X
      "yc":0.0, // Punkt srodkowy Y
      "xf":0.0, // Punkt koncowy X
      "yf":0.0, // Punkt koncowy Y
      "stops":  // Lista kolorow (pary [odleglosc, kolor])
                // Ogleglosc wzgledna z przedzialu (0;1)
        [
          [0,"#FFFFFF"],
          [1,"#000000"]
        ],
      "spread":0,
      "mode":0
    }
  "transform":[[1,0,0],[0,1,0],[0,0,1]] // Opcjonalne
}
\end{lstlisting}

Pełna implementacja za pomocą \emph{createRadialGradient}.

\item Gradient stożkowy
\begin{lstlisting}[language=JavaScript,numbers=none]	    "brush":			
{
  "style":17,
  "gradient":
    {  
      "type":2,
      "xc":0.0, // Punkt srodkowy X
      "yc":0.0, // Punkt srodkowy Y
      "a":0.0,  // Kat
      "stops":  // Lista kolorow (pary [odleglosc, kolor])
                // Ogleglosc wzgledna z przedzialu (0;1)
        [
          [0,"#FFFFFF"],
          [1,"#000000"]
        ],
      "spread":0,
      "mode":0
    }
  "transform":[[1,0,0],[0,1,0],[0,0,1]] // Opcjonalne
}
\end{lstlisting}

Brak implementacji w \emph{canvas}.

\item Tekstura
\begin{lstlisting}[language=JavaScript,numbers=none]
"brush":			
  {
    "style":24,
    "image":"data:image/png;base64,DIUSHFIUSHRIUDSHIFIUHI329859vdsy7vy87dv8sgv87sdgvgsd8gvyu)",
    "transform":[[1,0,0],[0,1,0],[0,0,1]] // Opcjonalne
  }
\end{lstlisting}

Implementacja częściowa za pomocą \emph{createPattern}, bez możliwości określenia transformacji.

\item Kolor
\begin{lstlisting}[language=JavaScript,numbers=none]	      	 "brush":			
  {
    "style":others,
    "color":"#FFFFFF"
    [,"transform":[[1,0,0],[0,1,0],[0,0,1]]]
  }
\end{lstlisting}

Pełna implementacja w \emph{canvas}.

\end{itemize}

\subsubsection{Pędzel krawędziowy}
Istnieje kilka rodzajów pędzli krawędziowych. Każdy z nich posiada inną strukturę przesyłanych danych, które przedstawiono poniżej. Wszystkie pędzle posiadają pewne cechy wspólne -- zakończenie linii, sposób łamania oraz posiadają określoną szerokość.
\begin{enumerate}
\item Zakończenia linii \emph{cap}\footnote{http://doc.qt.digia.com/qt/qt.html\#PenCapStyle-enum}:
\begin{itemize}
\item \emph{Qt::FlatCap} (wartość 0x00) --- proste ścięcie linią prostopadłą do stycznej na końcu krzywej,
\item \emph{Qt::SquareCap} (wartość 0x10) --- zakończenie kwadratowe, bardzo podobne do poprzedniego typu,
\item \emph{Qt::RoundCap} (wartość 0x20) --- gładkie, zaokrąglone zakończenie.
\end{itemize}
Wszystkie opcje dostępne w \emph{canvas}.

\item Złączenia (załamania) linii \emph{join}\footnote{http://doc.qt.digia.com/qt/qt.html\#PenJoinStyle-enum}:
\begin{itemize}
\item \emph{Qt::MiterJoin} (wartość 0x00) \footnote{Wraz z tą wartością parametru musi dodatkowo pojawić się parametr \emph{miter} określający promień zaokrąglania załamań linii.}
\item \emph{Qt::BevelJoin} (wartość 0x40)
\item \emph{Qt::RoundJoin} (wartość 0x80)
\item \emph{Qt::SvgMiterJoin} (wartość 0x100)
\end{itemize}
Opcja \emph{Qt::SvgMiterJoin} to wewnętrzny sposób \emph{Qt} złączania linii stosowany przy grafikach \emph{SVG}. Metoda ta nie jest dostępna w \emph{canvas}.

\item Szerokość linii \emph{width}

Wartość tego parametru stanowi szerokość linii mierzoną w pikselach.
\end{enumerate}
Pełne wsparcie w \emph{canvas} dla wszystkich atrybutów.

W celu wyczyszczenia wartości pędzla jako wartość dla klucza \emph{pen} wysyłany jest obiekt pusty.

\begin{itemize}
\item Kolor
\begin{lstlisting}[language=JavaScript,numbers=none]
"pen":
  {
    "color":"#FFFFFF",
    "cap":0,
    "join":1,
    "width":10
  }
\end{lstlisting}
Pełne wsparcie w \emph{canvas}.

\item Linia przerywana
\begin{lstlisting}[language=JavaScript,numbers=none]
"pen":
  {
    "color":"#FFFFFF",
    "dash":		// !!! optional
      {
        "offset":0.5,
        "pattern":[10,10,30,10,20,10]
      },
    "cap":0,
    "join":1,
    "width":10
  }
\end{lstlisting}
Tymczasowy brak wsparcia \emph{canvas} we wszystkich przeglądarkach. Możliwe jedynie symulowanie za pomocą pojedynczych kresek. Brak implementacji w projekcie, co skutkuje wyświetlaniem linii przerywanych jako linie ciągłe. Ten rodzaj pędzla jest używany bardzo rzadko w aplikacjach, dlatego nie stanowi to dużego problemu.


\item Linia z teksturą
\begin{lstlisting}[language=JavaScript,numbers=none]
"pen":
  {
    "brush": { /* Patrz opis pedzla wypelnienia */ },
    "cap":0,
    "join":0,
    "miter":0.5,
    "width":10
  }
\end{lstlisting}
Pełne wsparcie w \emph{canvas}.

\end{itemize}

\subsubsection{Czcionka}
Opis czcionki reprezentowany jest w formacie CSS \cite{cssfont} i obejmuje kolejno:
\begin{itemize}
\item styl,
\item wariant,
\item rozmiar,
\item wysokość,
\item rodzinę (lista nazw rozdzielona przecinkami).
\end{itemize}

Przykład:
\begin{lstlisting}[language=JavaScript,numbers=none]
"font":"italic small-caps lighter 15px Sans-Serif"
\end{lstlisting}

Pełne wsparcie w \emph{canvas}. 

\subsubsection{Metoda kompozycji}
Parametr ten określa w jaki sposób łączone są kolejne nakładające się warstwy.\cite{compositionmode} Metody dostępne w \emph{canvas}:
\begin{itemize}
\item \emph{source-atop},
\item \emph{source-in},
\item \emph{source-out},
\item \emph{source-over},
\item \emph{destination-atop},
\item \emph{destination-in},
\item \emph{destination-out},
\item \emph{destination-over},
\item \emph{lighter},
\item \emph{darker},
\item \emph{xor},
\item \emph{copy}.
\end{itemize}

Lista mozliwych wariantów po stronie aplikacji \emph{Qt} jest zdecydowanie dłuższa od wyżej przedstawionej, która obejmujenie jedynie kolejność i sposoby łaczenia poszczególnych warstw. W \emph{Qt} możliwe są również złożone metody mieszania kolorów warstw nie wspierane przez element \emph{canvas}, dlatego też przesyłane są tylko kompatybilne opcje, a wszystkie pozostałe zastępowane domyślną wartością \emph{source-atop}. Metody te są bardzo specyficzne i rzadko używane, dlatego brak ich implemetacnji nie wpływa na odbiór i użytkowanie typowych aplikacji użytkowych.

\subsubsection{Obcinanie}
\begin{lstlisting}[language=JavaScript,numbers=none]
"clip":
  {
     // Lista punktow sciezki ciecia
     "data":			
       [
         ["t":0,"p":[[0,0]]],                         // moveTo
         ["t":1,"p":[[10,10]]],                       // lineTo
         ["t":2,"p":[[10,10],[100,100]]],             // quadTo
         ["t":2,"p":[[10,10],[100,100],[1000,1000]]], // cubicTo
       ],
     "fill":0 // 0: Qt::OddEvenFill
              // 1: Qt::WindingFill
  }
\end{lstlisting}
Obcinanie polega na ograniczaniu obszaru rysowania poprzez dowolną krzywą. 
Pełne wsparcie w \emph{canvas}, jednak z powodu błędów w implementacji popularnych przeglądarek (\emph{Google Chrome} oraz \emph{Mozilla Firefox}) nie została ona zaimplementowana. Problem został opisany w podrozdziale \ref{sec:napotkane_problemy}.