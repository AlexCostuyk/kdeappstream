\subsection{Buforowanie grafik}
Większość współczesnych aplikacji z graficznym interfejsem użytkownika zawiera pewne elementy jak ikona lub obraz w formacie PNG (ang. Portable Network Graphics), których nie da się opisać za pomocą prostych elementów jak linia czy elipsa, lecz trzeba je przesłać w postaci danych binarnych. W przypadku takich aplikacji przy każdej zmianie widgeta zawierającego grafikę następuje ponowne jest narysowanie, a więc ta sama ikona czy obraz PNG będzie wykorzystywany wielokrotnie. 

Problem ten rozwiązano poprzez zastosowanie buforowania obrazów. Każdy taki element najpierw poddawany jest operacji wyznaczania wartości skrótowej (ang. hash value) z wykorzystaniem algorytmu MD5. Klucz ten przesyłany jest w klientowi zamiast właściwych danych obrazka, a sama grafika jest umieszczana w buforze serwera aplikacji. Klient chcąc narysować grafikę pobiera ją z serwera podając w zapytaniu otrzymany klucz. Po pobraniu obrazu serwer usuwa go z pamięci lecz nie zapomina o nim całkowicie. Zapamiętuje bowiem informację o tym, że dany obrazek został już przez klienta pobrany i od tej pory to klient jest odpowiedzialny za ponowne wykorzystanie pobranego wcześniej pliku. Serwer nie umieści ponownie w buforze obrazka o identycznej wartości skrótowej, dopóki klient nie poinformuje serwera, że nie przechowuje już dłużej danej grafiki i w przypadku jej ponownego wykorzystania serwer musi ponownie umieścić plik w buforze. 

Takie rozwiązanie zapewnia:
\begin{itemize}
\item zabezpieczenie przed zapełnieniem bufora serwera,
\item znaczne zmniejszenie transferu danych w aplikacjach wykorzystujących grafiki rastrowe --- przeprowadzone testy wykazały, że szybkość działania aplikacji wzrasta nawet dwukrotnie,
\item możliwość kontroli wielkości bufora przez klienta dzięki możliwości zdalnego czyszczenia pamięci kluczy.
\end{itemize}