Aby umożliwić renderowanie elementów po stronie klienta należało utworzyć wspólny format danych bazując na wejściu ze strony biblioteki Qt oraz potrzebnych danych wyjściowych dla obiektu Canvas w języku HTML5. 

Każda komenda rysowania po stronie klienta składa się z podstawowych informacji dotyczących rysowanego obiektu, takich jak: identyfikator, pozycja czy rozmiar, oraz listy prostych elementów z których danych obiekt jest złożony (linie, prostokąty, etc.). Poniżej przedstawiono format pojedyńczej komendy rysowania.


\begin{lstlisting}[language=JavaScript,caption=Komenda renderowania elementu interfejsu]
{
  "command":"draw",
  "widget":{
    "id": 12431,          // Identyfikator
    ["z": 0,]             // Pozycja na stosie obiektow potomnych
    "name":"QLineEdit",   // Nazwa obiektu
    "flags": 0x1029,      // Flagi obiektu (definiuja jego typ 
                          // i wlasciwosci)
    "x": 100,             // Pozycja X
    "y": 120,             // Pozycja Y
    "w": 200,             // Szerokosc
    "h": 150,             // Wysokosc
    "r":{                 // Renderowany obszar elementu
      "x": 0,             // Pozycja obszaru X
      "y": 0,             // Pozycja obszaru Y
      "w": 200,           // Szerokosc obszaru
      "h": 150,           // Wysokosc obszaru
    }
  },
  "render":[]             // Lista elementow do narysowania
}
\end{lstlisting}

Poniżej w kolejności alfabetycznej przedstawiono elementy, z których może być zbudowany każdy obiekt klasy QWidget. Każdy opis zawiera deklarację metody podklasy QPaintEngine wykorzystywanej po stronie serwera, format przesyłanych danych oraz sposób interpretacji tych danych po stronie klienta.

\subsection{Elipsy}
\begin{lstlisting}[language=C++,numbers=none]
virtual void QPaintEngine::drawEllipse( const QRectF & rect );
virtual void QPaintEngine::drawEllipse( const QRect & rect );
\end{lstlisting}
\begin{lstlisting}[language=JavaScript,numbers=none]
{
  "t":"ellipse",
  "x":0.0,       // Pozycja srodka X
  "y":0.0,       // Pozycja srodka Y
  "w":10.0,      // Srednica pozioma
  "h":10.0       // Srednica pionowa
}
\end{lstlisting}

\subsection{Kwadraty}
\begin{lstlisting}[language=C++,numbers=none]
virtual void QPaintEngine::drawRects( const QRectF * rects, 
                                      int rectCount );
virtual void QPaintEngine::drawRects( const QRect * rects, 
                                      int rectCount );
\end{lstlisting}
\begin{lstlisting}[language=JavaScript,numbers=none]
{
  "t":"rect",
  "x":0.0,      // Pozycja lewego-gornego wierzcholka X
  "y":0.0,      // Pozycja lewego-gornego wierzcholka Y
  "w":10.0,     // Szerokosc
  "h":10.0      // Wysokosc
}
\end{lstlisting}

\subsection{Linie}
\begin{lstlisting}[language=C++,numbers=none]
virtual void QPaintEngine::drawLines( const QLineF * lines, 
                                      int lineCount );
virtual void QPaintEngine::drawLines( const QLine * lines, 
                                      int lineCount );
\end{lstlisting}
\begin{lstlisting}[language=JavaScript,numbers=none]
{
  "t":"line",
  "xs":0.0,    // Pozycja startowa X
  "ys":0.0,    // Pozycja startowa Y
  "xe":10.0,   // Pozycja koncowa X
  "ye":10.0    // Pozycja koncowa Y
}
\end{lstlisting}

\subsection{Obrazy}
\begin{lstlisting}[language=C++,numbers=none]
virtual void QPaintEngine::drawImage( const QRectF & rectangle, 
                                      const QImage & image, 
                                      const QRectF & sr, 
                                      Qt::ImageConversionFlags flags = Qt::AutoColor );
virtual void QPaintEngine::drawPixmap( const QRectF & r, 
                                       const QPixmap & pm, 
                                       const QRectF & sr );
virtual void QPaintEngine::drawTiledPixmap( const QRectF & rect, 
                                            const QPixmap & pixmap, 
                                            const QPointF & p );
\end{lstlisting}
\begin{lstlisting}[language=JavaScript,numbers=none]
{
  "t":"image",
  "data":"Ja8SA9c72b71HDj8", // Identyfikator obrazu
  "x":0.0,                   // Pozycja X
  "y":0.0                    // Pozycja Y
}
\end{lstlisting}

Pełna implementacja, natywnie wspierane w \emph{canvas} za pomocą metody drawImage.

\subsection{Wielokąty}
\begin{lstlisting}[language=C++,numbers=none]
virtual void QPaintEngine::drawPolygon( const QPointF * points, 
                                        int pointCount, 
                                        PolygonDrawMode mode );
virtual void QPaintEngine::drawPolygon( const QPoint * points, 
                                        int pointCount, 
                                        PolygonDrawMode mode );
\end{lstlisting}
\begin{lstlisting}[language=JavaScript,numbers=none]
{
  "t":"polygon",
  "mode":0, // 0: QPaintEngine::OddEvenMode
            // 1: QPaintEngine::WindingMode
            // 2: QPaintEngine::ConvexMode
            // 3: QPaintEngine::PolylineMode	
            // http://doc.qt.digia.com/stable/qpaintengine.html#PolygonDrawMode-enum
  "data":   // Lista punktow do polaczenia
    [
      [0.0,0.0],
      [10.0,10.0],
      [123.0,123.0]
    ]
}
\end{lstlisting}

Pełna implementacja, natywnie wspierane w \emph{canvas} za pomocą metod \emph{moveTo}, \emph{lineTo} oraz \emph{closePath}.

\subsection{Punkty}

\begin{lstlisting}[language=C++,numbers=none]
virtual void QPaintEngine::drawPoints( const QPointF * points, 
                                       int pointCount );
virtual void QPaintEngine::drawPoints( const QPoint * points, 
                                       int pointCount );
\end{lstlisting}
\begin{lstlisting}[language=JavaScript,numbers=none]
{
  "t":"points",
  "data":              // Lista punktow
    [
      [0.0,0.0],
      [10.0,10.0],
      [123.0,123.0]
    ]
},
\end{lstlisting}

Pełna implementacja, natywnie wspierane w \emph{canvas}, za pomocą \emph{strokeRect} rysowany jest kwadrat o rozmiarach 1 na 1 piksel.

\subsection{Ścieżki}
\begin{lstlisting}[language=C++,numbers=none]
virtual void QPaintEngine::drawPath( const QPainterPath & path );
\end{lstlisting}
\begin{lstlisting}[language=JavaScript,numbers=none]
{
  "t":"path",
  "data":          // Lista punktow
    [
      ["t":0,"p":[[0,0]]],      // moveTo
      ["t":1,"p":[[10,10]]],    // lineTo
      ["t":2,"p":[[10,10],[100,100]]],  // quadTo
      ["t":2,"p":[[10,10],[100,100],[1000,1000]]],  // cubicTo
    ],
  "fill":0 // 0: Qt::OddEvenFill
           // 1: Qt::WindingFill
           // http://doc.qt.digia.com/stable/qt.html#FillRule-enum
}
\end{lstlisting}

Pełna implementacja, natywnie wspierane w \emph{canvas} za pomocą metod \emph{lineTo}, \emph{lineTo}, \emph{quadraticCurveTo} oraz \emph{bezierCurveTo}.

\subsection{Tekst}
\begin{lstlisting}[language=C++,numbers=none]
virtual void QPaintEngine::drawTextItem( const QPointF & p, 
                                         const QTextItem & textItem );
\end{lstlisting}
\begin{lstlisting}[language=JavaScript,numbers=none]
{
  "t":"text",
  "data":
  {
    "text":"Przykladowy tekst",     // Tekst w kodowaniu UTF8
    "ascent":0,                     // Dystans od linii bazowej do 
                                    // najwyzej polozonego punktu
    "descent":0,                    // Dystans od linii bazowej do 
                                    // najnizej polozonego punktu
    "x":0,                          // Pozycja X
    "y":0,                          // Pozycja Y
    "font":"CSS-format font string" // Informacje o czcionce 
                                    // w formacie CSS
  }
}
\end{lstlisting}