Renderowanie jest operacją realizowaną za każdym razem gdy wykryta zostania zmiana dowolnego elementu interfejsu użytkownika. W standardowej aplikacji do kolejki zdarzeń trafia informacja o konieczności odświeżenia widoku aby w kolejnym kroku narysować wybrany element na ekranie. 

W projekcie ze względu na konieczność rysowania interfejsu aplikacji na zdalnej maszynie konieczne jest przechwycenie informacji z kolejki zdarzeń o zmianie parametrów widgeta oraz utworzenie opisu jego wyglądu w wykorzystywanym formacie wymiany informacji. Szczegóły struktur danych, ich opis oraz sposoby przetwarzania zostały opisane w sekcji \ref{data_protocol}.

Biblioteka \emph{Qt} nie umożliwia pobrania informacji o wyglądzie elementu bezpośrednio. Należało więc wykorzystać wewnętrzny mechanizm rysowania. Dokonano to poprzez zaimplementowanie dwóch klas dziedziczących po \emph{QPaintDevice} oraz \emph{QPaintEngine}. 

Klasa \emph{QPaintDevice} opisuje wirtualne urządzenie, na którym dokonywane jest renderowanie elementu. Jej implementacja w projekcie nie zawiera żadnej logiki. Informuje jedynie pozostałe elementy systemu o typie, maksymalnym rozmiarze i stosowanej jednostce miary wirtualnego urządzenia. 

Klasa \emph{QPaintEngine}, zgodnie z założeniami biblioteki \emph{Qt}, jest odpowiedzialna za rysowanie na obiekcie \emph{QPaintDevice}. Udostępnia więc szereg metod dokonujących rysowania podstawowych elementów takich jak linie, okręgi czy krzywe oraz umożliwia ustawienie palety kolorów i przekształceń geometrycznych. W projekcie metody te zaimplementowano w taki sposób aby zapisywały informacje o rysowanych elementach w formacie dnych do przesłania klientowi. Informacje te są przetrzymywane w buforze a po jego zapełnieniu są przekazywane klientowi i w rzeczywistości elementy nie są w ogóle rysowane po stronie serwera. Poniżej przedstawiono nagłówek klasy \emph{QPaintEngine} zawierającą listę wszystkich dostępnych metod.

\begin{lstlisting}[language=C++,numbers=left,caption=Nagłówek klasy \emph{QPaintEngine}]
class PaintEngine : public QPaintEngine
{
    public:

        explicit PaintEngine();
        virtual ~PaintEngine();
        virtual bool begin(QPaintDevice * pdev);
        virtual void drawEllipse(const QRectF & rect);
        virtual void drawEllipse(const QRect & rect);
        virtual void drawImage(const QRectF & rectangle, 
                               const QImage & image, 
                               const QRectF & sr,
                               Qt::ImageConversionFlags flags 
                                      = Qt::AutoColor);
        virtual void drawLines(const QLineF * lines, int lineCount);
        virtual void drawLines(const QLine * lines, int lineCount);
        virtual void drawPath(const QPainterPath & path);
        virtual void drawPixmap(const QRectF & r, 
                                const QPixmap & pm, 
                                const QRectF & sr);
        virtual void drawPoints(const QPointF * points, 
                                int pointCount);
        virtual void dawPoints(const QPoint * points, 
                                int pointCount);
        virtual void drawPolygon(const QPointF * points, 
                                 int pointCount, 
                                 PolygonDrawMode mode);
        virtual void drawPolygon(const QPoint * points, 
                                 int pointCount, 
                                 PolygonDrawMode mode);
        virtual void drawRects (const QRectF * rects, int rectCount);
        virtual void drawRects (const QRect * rects, int rectCount);
        virtual void drawTextItem(const QPointF & p, 
                                  const QTextItem & textItem);
        virtual void drawTiledPixmap(const QRectF & rect, 
                                     const QPixmap & pixmap, 
                                     const QPointF & p);
        virtual bool end();
        virtual Type type() const;
        virtual void updateState (const QPaintEngineState & state);
};
\end{lstlisting}