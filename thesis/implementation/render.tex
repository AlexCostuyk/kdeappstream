TODO
\begin{itemize}
\item{OPIS KLASY WebRenderer}
\item{POJEDYŃCZY CYKL RENDEROWANIA.}
\item Opis wykorzystania danych pod każdą sekcją!
\end{itemize}

Aby umożliwić renderowanie elementów po stronie klienta należało utworzyć wspólny format danych bazując na wejściu ze strony biblioteki Qt oraz potrzebnych danych wyjściowych dla obiektu Canvas w języku HTML5. 

Każda komenda rysowania po stronie klienta składa się z podstawowych informacji dotyczących rysowanego obiektu, takich jak: identyfikator, pozycja czy rozmiar, oraz listy prostych elementów z których danych obiekt jest złożony (linie, prostokąty, etc.). Poniżej przedstawiono format pojedyńczej komendy rysowania wraz z wszystkimi możliwymi elementami budującymi widgety w aplikacjach opartych o bibliotekę \emph{Qt}.


\begin{lstlisting}[language=JavaScript,numbers=none,caption=Komenda renderowania elementu interfejsu]
{
  "command":"draw",
  "widget":{
    "id": 12431,          // Identyfikator
    ["z": 0,]             // Pozycja na stosie obiektow potomnych
    "name":"QLineEdit",   // Nazwa obiektu
    "flags": 0x1029,      // Flagi obiektu (definiuja jego typ 
                          // i wlasciwosci)
    "x": 100,             // Pozycja X
    "y": 120,             // Pozycja Y
    "w": 200,             // Szerokosc
    "h": 150,             // Wysokosc
    "r":{                 // Renderowany obszar elementu
      "x": 0,             // Pozycja obszaru X
      "y": 0,             // Pozycja obszaru Y
      "w": 200,           // Szerokosc obszaru
      "h": 150,           // Wysokosc obszaru
    }
  },
  "render":[]             // Lista elementow do narysowania
}
\end{lstlisting}

Poniżej w kolejności alfabetycznej przedstawiono elementy, z których może być zbudowany każdy obiekt klasy \emph{QWidget}. Każdy opis zawiera deklarację metody podklasy \emph{QPaintEngine} wykorzystywanej po stronie serwera, format przesyłanych danych oraz opis metody interpretacji tych danych po stronie klienta.

\subsection{Elipsy}
\begin{lstlisting}[language=C++,numbers=none]
virtual void QPaintEngine::drawEllipse( const QRectF & rect );
virtual void QPaintEngine::drawEllipse( const QRect & rect );
\end{lstlisting}
\begin{lstlisting}[language=JavaScript,numbers=none]
{
  "t":"ellipse",
  "x":0.0,       // Pozycja srodka X
  "y":0.0,       // Pozycja srodka Y
  "w":10.0,      // Srednica pozioma
  "h":10.0       // Srednica pionowa
}
\end{lstlisting}

\subsection{Kwadraty}
\begin{lstlisting}[language=C++,numbers=none]
virtual void QPaintEngine::drawRects( const QRectF * rects, 
                                      int rectCount );
virtual void QPaintEngine::drawRects( const QRect * rects, 
                                      int rectCount );
\end{lstlisting}
\begin{lstlisting}[language=JavaScript,numbers=none]
{
  "t":"rect",
  "x":0.0,      // Pozycja lewego-gornego wierzcholka X
  "y":0.0,      // Pozycja lewego-gornego wierzcholka Y
  "w":10.0,     // Szerokosc
  "h":10.0      // Wysokosc
}
\end{lstlisting}

\subsection{Linie}
\begin{lstlisting}[language=C++,numbers=none]
virtual void QPaintEngine::drawLines( const QLineF * lines, 
                                      int lineCount );
virtual void QPaintEngine::drawLines( const QLine * lines, 
                                      int lineCount );
\end{lstlisting}
\begin{lstlisting}[language=JavaScript,numbers=none]
{
  "t":"line",
  "xs":0.0,    // Pozycja startowa X
  "ys":0.0,    // Pozycja startowa Y
  "xe":10.0,   // Pozycja koncowa X
  "ye":10.0    // Pozycja koncowa Y
}
\end{lstlisting}

\subsection{Obrazy}
\begin{lstlisting}[language=C++,numbers=none]
virtual void QPaintEngine::drawImage( const QRectF & rectangle, 
                                      const QImage & image, 
                                      const QRectF & sr, 
                                      Qt::ImageConversionFlags flags = Qt::AutoColor );
virtual void QPaintEngine::drawPixmap( const QRectF & r, 
                                       const QPixmap & pm, 
                                       const QRectF & sr );
virtual void QPaintEngine::drawTiledPixmap( const QRectF & rect, 
                                            const QPixmap & pixmap, 
                                            const QPointF & p );
\end{lstlisting}
\begin{lstlisting}[language=JavaScript,numbers=none]
{
  "t":"image",
  "data":"Ja8SA9c72b71HDj8", // Identyfikator obrazu
  "x":0.0,                   // Pozycja X
  "y":0.0                    // Pozycja Y
}
\end{lstlisting}

Pełna implementacja, natywnie wspierane w \emph{canvas} za pomocą metody drawImage.

\subsection{Wielokąty}
\begin{lstlisting}[language=C++,numbers=none]
virtual void QPaintEngine::drawPolygon( const QPointF * points, 
                                        int pointCount, 
                                        PolygonDrawMode mode );
virtual void QPaintEngine::drawPolygon( const QPoint * points, 
                                        int pointCount, 
                                        PolygonDrawMode mode );
\end{lstlisting}
\begin{lstlisting}[language=JavaScript,numbers=none]
{
  "t":"polygon",
  "mode":0, // 0: QPaintEngine::OddEvenMode
            // 1: QPaintEngine::WindingMode
            // 2: QPaintEngine::ConvexMode
            // 3: QPaintEngine::PolylineMode	
            // http://doc.qt.digia.com/stable/qpaintengine.html#PolygonDrawMode-enum
  "data":   // Lista punktow do polaczenia
    [
      [0.0,0.0],
      [10.0,10.0],
      [123.0,123.0]
    ]
}
\end{lstlisting}

Pełna implementacja, natywnie wspierane w \emph{canvas} za pomocą metod \emph{moveTo}, \emph{lineTo} oraz \emph{closePath}.

\subsection{Punkty}

\begin{lstlisting}[language=C++,numbers=none]
virtual void QPaintEngine::drawPoints( const QPointF * points, 
                                       int pointCount );
virtual void QPaintEngine::drawPoints( const QPoint * points, 
                                       int pointCount );
\end{lstlisting}
\begin{lstlisting}[language=JavaScript,numbers=none]
{
  "t":"points",
  "data":              // Lista punktow
    [
      [0.0,0.0],
      [10.0,10.0],
      [123.0,123.0]
    ]
},
\end{lstlisting}

Pełna implementacja, natywnie wspierane w \emph{canvas}, za pomocą \emph{strokeRect} rysowany jest kwadrat o rozmiarach 1 na 1 piksel.

\subsection{Ścieżki}
\begin{lstlisting}[language=C++,numbers=none]
virtual void QPaintEngine::drawPath( const QPainterPath & path );
\end{lstlisting}
\begin{lstlisting}[language=JavaScript,numbers=none]
{
  "t":"path",
  "data":          // Lista punktow
    [
      ["t":0,"p":[[0,0]]],      // moveTo
      ["t":1,"p":[[10,10]]],    // lineTo
      ["t":2,"p":[[10,10],[100,100]]],  // quadTo
      ["t":2,"p":[[10,10],[100,100],[1000,1000]]],  // cubicTo
    ],
  "fill":0 // 0: Qt::OddEvenFill
           // 1: Qt::WindingFill
           // http://doc.qt.digia.com/stable/qt.html#FillRule-enum
}
\end{lstlisting}

Pełna implementacja, natywnie wspierane w \emph{canvas} za pomocą metod \emph{lineTo}, \emph{lineTo}, \emph{quadraticCurveTo} oraz \emph{bezierCurveTo}.

\subsection{Tekst}
\begin{lstlisting}[language=C++,numbers=none]
virtual void QPaintEngine::drawTextItem( const QPointF & p, 
                                         const QTextItem & textItem );
\end{lstlisting}
\begin{lstlisting}[language=JavaScript,numbers=none]
{
  "t":"text",
  "data":
  {
    "text":"Przykladowy tekst",     // Tekst w kodowaniu UTF8
    "ascent":0,                     // Dystans od linii bazowej do 
                                    // najwyzej polozonego punktu
    "descent":0,                    // Dystans od linii bazowej do 
                                    // najnizej polozonego punktu
    "x":0,                          // Pozycja X
    "y":0,                          // Pozycja Y
    "font":"CSS-format font string" // Informacje o czcionce 
                                    // w formacie CSS
  }
}
\end{lstlisting}

\subsection{Stan pędzli}
\begin{lstlisting}[language=C++,numbers=none]
virtual void QPaintEngine::updateState(const QPaintEngineState& state)
\end{lstlisting}
\begin{lstlisting}[language=JavaScript,numbers=none]
{
  "t":"state",
  "data":
    {
      // Opis pedzla krawedzi
      "pen": { /* ... */ },
      
      // Opis pedzla wypelnienia 
      "brush": { /* ... */ },
      
      // Czcionka
      "font": "Opis czcionki w formacie CSS",

      // Offset pedzla
      "brushorigin":
        {
          "x":0.0,
          "y":0.0
        },

      // Macierz transformacji 3x3
      "transform":[[1,0,0],[0,1,0],[0,0,1]], 
      
      // Metoda kompozycji
      "composition":"source-over",
      
      // Opis pedzla tla
      "bbrush": { /* ... */ },
      
      // Przezroczystosc
      "opacity":0.5,
      
      // Obcinanie
      "clip": { /* ... */ }
    }
}
\end{lstlisting}

\subsubsection{Pędzele wypełnienia i tła}

\begin{itemize}
\item Brak wypełnienia
\begin{lstlisting}[language=JavaScript,numbers=none]	    "brush":
  {
    "style":0,
  }
\end{lstlisting}

Pełna implementacja za pomocą ustawienia koloru na RGB(0,0,0,0) -- całkowicie przeźroczysty.

\item Gradient liniowy
\begin{lstlisting}[language=JavaScript,numbers=none]	    "brush":			
{
  "style":15,
  "gradient":
    {  
      "type":0,
      "xs":0.0, // Punkt poczatkowy X
      "ys":0.0, // Punkt poczatkowy Y
      "xe":0.0, // Punkt koncowy X
      "ye":0.0, // Punkt koncowy Y
      "stops":  // Lista kolorow (pary [odleglosc, kolor])
                // Ogleglosc wzgledna z przedzialu (0;1)
        [
          [0,"#FFFFFF"],
          [1,"#000000"]
        ],
      "spread":0, // Wypelnienie poza obszarem gradientu
                  // 0: QGradient::PadSpread
                  // 1: QGradient::ReflectSpread
                  // 2: QGradient::RepeatSpread
      "mode":0    // Definiuje sposob interpretacji wspolrzednych
                  // 0: QGradient::LogicalMode
                  // 1: QGradient::StretchToDeviceMode
                  // 2: QGradient::ObjectBoundingMode
    }
  "transform":[[1,0,0],[0,1,0],[0,0,1]] // Opcjonalne
}
\end{lstlisting}

Pełna implementacja za pomocą \emph{createLinearGradient}.

\item Gradient kołowy
\begin{lstlisting}[language=JavaScript,numbers=none]	    "brush":			
{
  "style":16,
  "gradient":
    {  
      "type":1,
      "xc":0.0, // Punkt srodkowy X
      "yc":0.0, // Punkt srodkowy Y
      "xf":0.0, // Punkt koncowy X
      "yf":0.0, // Punkt koncowy Y
      "stops":  // Lista kolorow (pary [odleglosc, kolor])
                // Ogleglosc wzgledna z przedzialu (0;1)
        [
          [0,"#FFFFFF"],
          [1,"#000000"]
        ],
      "spread":0,
      "mode":0
    }
  "transform":[[1,0,0],[0,1,0],[0,0,1]] // Opcjonalne
}
\end{lstlisting}

Pełna implementacja za pomocą \emph{createRadialGradient}.

\item Gradient stożkowy
\begin{lstlisting}[language=JavaScript,numbers=none]	    "brush":			
{
  "style":17,
  "gradient":
    {  
      "type":2,
      "xc":0.0, // Punkt srodkowy X
      "yc":0.0, // Punkt srodkowy Y
      "a":0.0,  // Kat
      "stops":  // Lista kolorow (pary [odleglosc, kolor])
                // Ogleglosc wzgledna z przedzialu (0;1)
        [
          [0,"#FFFFFF"],
          [1,"#000000"]
        ],
      "spread":0,
      "mode":0
    }
  "transform":[[1,0,0],[0,1,0],[0,0,1]] // Opcjonalne
}
\end{lstlisting}

Brak implementacji w \emph{canvas}.

\item Tekstura
\begin{lstlisting}[language=JavaScript,numbers=none]
"brush":			
  {
    "style":24,
    "image":"data:image/png;base64,DIUSHFIUSHRIUDSHIFIUHI329859vdsy7vy87dv8sgv87sdgvgsd8gvyu)",
    "transform":[[1,0,0],[0,1,0],[0,0,1]] // Opcjonalne
  }
\end{lstlisting}

Implementacja częściowa za pomocą \emph{createPattern}, bez możliwości określenia transformacji.

\item Kolor
\begin{lstlisting}[language=JavaScript,numbers=none]	      	 "brush":			
  {
    "style":others,
    "color":"#FFFFFF"
    [,"transform":[[1,0,0],[0,1,0],[0,0,1]]]
  }
\end{lstlisting}

Pełna implementacja w \emph{canvas}.

\end{itemize}

\subsubsection{Pędzel krawędziowy}
Istnieje kilka rodzajów pędzli krawędziowych. Każdy z nich posiada inną strukturę przesyłanych danych, które przedstawiono poniżej. W tym miejscu należy również zaznaczyć kilka opcji wspólnych, występujące we wszystkich rodzajach pędzli:
\begin{enumerate}
\item Zakończenia linii \emph{cap}\footnote{http://doc.qt.digia.com/qt/qt.html\#PenCapStyle-enum}
\begin{itemize}
\item \emph{Qt::FlatCap} (wartość 0x00) --- proste ścięcie linią prostopadłą do stycznej na końcu krzywej,
\item \emph{Qt::SquareCap} (wartość 0x10) --- zakończenie kwadratowe, bardzo podobne do poprzedniego typu,
\item \emph{Qt::RoundCap} (wartość 0x20) --- gładkie, zaokrąglone zakończenie.
\end{itemize}
Wszystkie opcje dostępne w \emph{canvas}.

\item Złączenia (załamania) linii \emph{join}\footnote{http://doc.qt.digia.com/qt/qt.html\#PenJoinStyle-enum}
\begin{itemize}
\item \emph{Qt::MiterJoin} (wartość 0x00) \footnote{Wraz z tą wartością parametru musi dodatkowo pojawić się parametr \emph{miter} określający promień zaokrąglania załamań linii.}
\item \emph{Qt::BevelJoin} (wartość 0x40)
\item \emph{Qt::RoundJoin} (wartość 0x80)
\item \emph{Qt::SvgMiterJoin} (wartość 0x100)
\end{itemize}
Opcja \emph{Qt::SvgMiterJoin} to wewnętrzny sposób \emph{Qt} złączania linii stosowany przy grafikach \emph{SVG}. Metoda ta nie jest dostępna w \emph{canvas}.

\item Szerokość linii \emph{width}

Wartość tego parametru stanowi szerokość linii mierzoną w pikselach.
\end{enumerate}
Pełne wsparcie w \emph{canvas}.

W celu wyczyszczenia wartości pędzla jako wartość dla klucza \emph{pen} wysyłany jest obiekt pusty.

\begin{itemize}
\item Kolor
\begin{lstlisting}[language=JavaScript,numbers=none]
"pen":
  {
    "color":"#FFFFFF",
    "cap":0,
    "join":1,
    "width":10
  }
\end{lstlisting}
Pełne wsparcie w \emph{canvas}.

\item Linia przerywana
\begin{lstlisting}[language=JavaScript,numbers=none]
"pen":
  {
    "color":"#FFFFFF",
    "dash":		// !!! optional
      {
        "offset":0.5,
        "pattern":[10,10,30,10,20,10]
      },
    "cap":0,
    "join":1,
    "width":10
  }
\end{lstlisting}
Brak wsparcia \emph{canvas}. Możliwe jedynie symulowanie za pomocą pojedynczych kresek. Brak implementacji w projekcie.

\item Linia z teksturą
\begin{lstlisting}[language=JavaScript,numbers=none]
"pen":
  {
    "brush": { /* Patrz opis pedzla wypelnienia */ },
    "cap":0,
    "join":0,
    "miter":0.5,
    "width":10
  }
\end{lstlisting}
Pełne wsparcie w \emph{canvas}.

\end{itemize}

\subsubsection{Czcionka}
Opis czcionki reprezentowany jest w formacie CSS\footnote{http://www.w3schools.com/cssref/pr\_font\_font.asp} i obejmuje kolejno:
\begin{itemize}
\item Styl
\item Wariant
\item Rozmiar
\item Wysokość
\item Rodzinę (lista nazw rozdzielona przecinkami)
\end{itemize}

Przykład:
\begin{lstlisting}[language=JavaScript,numbers=none]
"font":"italic small-caps lighter 15px Sans-Serif"
\end{lstlisting}

Pełne wsparcie w \emph{canvas}. 

\subsubsection{Metoda kompozycji}
Parametr ten określa w jaki sposób łączone są kolejne nakładające się warstwy.\footnote{http://doc.qt.digia.com/qt/qpainter.html\#CompositionMode-enum}
\begin{itemize}
\item \emph{source-atop}
\item \emph{source-in}
\item \emph{source-out}
\item \emph{source-over}
\item \emph{destination-atop}
\item \emph{destination-in}
\item \emph{destination-out}
\item \emph{destination-over}
\item \emph{lighter}
\item \emph{darker}
\item \emph{xor}
\item \emph{copy}
\end{itemize}

Lista mozliwych wariantów po stronie serwera jest zdecydowanie dłuższa od wyżej przedstawionej, która obejmujenie jedynie kolejność i sposoby łaczenia poszczególnych warstw. Po stronie serwera możliwe są również złożone metody mieszania kolorów warstw nie wspierane przez element \emph{canvas}, dlatego też przesyłane są tylko kompatybilne opcje, a wszystkie pozostałe zastępowane domyślną wartością \emph{source-atop}.

\subsubsection{Obcinanie}
\begin{lstlisting}[language=JavaScript,numbers=none]
"clip":
  {
     // Lista punktow sciezki ciecia
     "data":			
       [
         ["t":0,"p":[[0,0]]],                         // moveTo
         ["t":1,"p":[[10,10]]],                       // lineTo
         ["t":2,"p":[[10,10],[100,100]]],             // quadTo
         ["t":2,"p":[[10,10],[100,100],[1000,1000]]], // cubicTo
       ],
     "fill":0 // 0: Qt::OddEvenFill
              // 1: Qt::WindingFill
  }
\end{lstlisting}
Obcinanie polega na ograniczaniu obszaru rysowania poprzez dowolną krzywą. 
Pełne wsparcie w \emph{canvas}, jednak z powodu błędów w implementacji popularnych przeglądarek (\emph{Google Chrome} oraz \emph{Mozilla Firefox}) nie została ona zaimplementowana. Problem został opisany w podrozdziale \ref{sec:napotkane_problemy}.