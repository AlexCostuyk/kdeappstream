\subsubsection{Rysowanie za pomocą zdarzeń, synchronizacja i buforowanie.}
\label{rendering_events}

Rysowanie widgetów we frameworku \emph{Qt} realizowane jest wewnątrz kolejki zdarzeń aplikacji. Zdarzenie rysowania powiadamia element interfejsu o konieczności przerysowania. Widget posiada wskaźnik do miejsca w pamięci gdzie powinien przeprowadzic operację renderowania, gdyż sam jest implementacją klasy QPaintDevice. 

Powyższy schemat działania wymagał znalezienia sposobu na zmuszenie widgetów do rysowania za pomocą specjalnie przygotowanej implementacji klasy QPaintEngine oraz QPaintDevice. Do tego celu wykorzystana została metoda biblioteki Qt:

\begin{lstlisting}[language=C++,numbers=none]
void QWidget::render(QPainter * painter, 
                     const QPoint & targetOffset = QPoint(), 
                     const QRegion & sourceRegion = QRegion(), 
                     RenderFlags renderFlags 
                          = RenderFlags(DrawWindowBackground | 
                                        DrawChildren))
\end{lstlisting}

Umożliwiła nam ona wskazanie obiektu QPainter wykorzystującego mechanizm renderowania serwera, tj. reimplementacje klasy QPaintEngine oraz QPaintDevice. Wykorzystanie tej metody powoduje wygenerowanie kolejnego zdarzenia i umieszczenie go w kolejce aplikacji. Powodowało to problem wpadania serwera w nieskończoną pętlę i uniemożliwiało jego dalsze poprawne funkcjonowanie. 

Rozwiązaniem problemu było stworzenie własnej kolejki widgetów, które wymagają renderowania. Kolejka ta jest opróżniana w pewnych odstępach czasu nie krótszych niż 100 milisekund. Wartość ta została dobrana eksperymentalnie tak aby uzyskać efekt płynnej interakcji z aplikacją.

\subsubsection{Znikający \emph{focus} okna aplikacji.}
\label{problems_focus}

Biblioteka \emph{Qt} stanowi niejako nakładkę dla natywnego zarządcy okien systemu operacyjnego. W zwizku z tym o kolejności okien na stosie decyduje system operacyjny. \emph{Qt} podejmuje jedynie odpowiednie czynności w celu aktualizacji graficznego interfejsu aplikacji w zależności od aktualnego stanu konkretnych okien definiowanego przez system operacyjny. W sytuacji kiedy użytkownik nie prowadzi żadnej intrakcji z systemem operacyjnym a jedynie z aplikacją, mogło by się zdarzyć, że niektóe zdarzenia było by ignorowane przez aplikację. Przykładem takiej sytuacji jest wprowadzanie tekstu na klawiaturze. \emph{Qt} przesyła zdarzenia klawiatury do widgeta, który aktualnie posiada focus w oknie, które znajduje się na szczycie stosu okien w systemie operacyjnym. W momencie kiedy dwóch zdalnych użytkowników uruchomiło by aplikację na tym samym serwerze, zdarzenia jednego użytkownika były by ignorowane ponieważ tylko jedno okno jednej aplikacji może być na szczycie stosu. 

Rozwiązaniem tego problemu była reimplementacja odpowiednich metod \emph{Qt} w celu symulacji zachowania stosu okien systemu operacyjnego wewnątrz samej aplikacji. W rezultacie aplikacja działa tak jakby zawsze była aktywna, dzięki czemu framework \emph{Qt} poprawnie renderuje wszystkie elementy interfejsu użytkownika.

\subsubsection{Kody znaków klawiatury.}
\label{problems_keyboard}

\emph{Qt} dostarcza platformowo niezależny opis kodów znaków klawiatury za pomocą zdefiniowanego typu wyliczeniowego \emph{Qt::Key}\footnote{http://doc.qt.digia.com/qt/qt.html\#Key-enum}. Niestety przeglądarki nie są dobrze ustandaryzowane i ich numeracja klawiszy znacznie różni się nie tylko między samymi platformami ale również między ich wersjami. Dodatkowo przeglądarki często nie wspierają wszystkich klawiszy przez to zakres kodów znaków jest inny niż w przypadku biblioteki \emph{Qt}.

Powyższe problemy wymusiły konieczność zaimplementowania metody konwertującej po stronie klienta kody klawiszy na odpowiadające im wartości typu wyliczeniowego \emph{Qt::Key}. Kod metody znajduje się w dodatkach (\ref{lst:addons_keyboard_method})

\subsubsection{Niedopasowanie czcionek.}
\label{problems_fonts}

Wiele aplikacji wykorzystuje niestandardowe czcionki, które nie są powszechnie dostępne na większości systemach operacyjncyh. Problem jest o tyle istotny, że w przypadku braku odpowiedniej czcionki po stronie klienta tekst (np. na przyciskach lub menu podręcznym) będzie zbyt mały aby go przeczytać lub zbyt duży aby w całości zmieścić się w obszarze elementu. 

Problem ten został rozwiązany w dwojaki sposób. Po pierwsze serwer przesyła listę rodzin czcionek, które mogą zostać wykorzystane jako alternatywy. Po drugie serwer dokonuje pomiar tekstu po jego wyrenderowaniu i przesyła tą informację do klienta gdzie wynikowy tekst moze zostać dodatkowo przeskalowany w celu osiągnięcia maksymalnej zgodności.

\subsubsection{Problemy z obcinaniem (ang. clipping) w przeglądarkach.}
\label{problems_clipping}
Podczas implementacji obcinania (ang. clipping) w kliencie rozpoznano problem z implementacją tej funkcjonalności w dwóch testowanych przeglądarkach -- \emph{Mozilla Firefox} w wersji 16.0.2 oraz \emph{Google Chrome} w wersji 23.0.1271.95 m. Problem jest znany i zgłoszony\footnote{http://code.google.com/p/chromium/issues/detail?id=135559}.

\subsubsection{Synchroniczne ładowanie obrazków}

Wszystkie grafiki używane do renderowania w elementach \emph{canvas} wymagają pełnego załadowania ich do pamięci przeglądarki. Język \emph{Javascript} nie pozwala na przeprowadzenie tej operacji synchroniczne. Pełny obraz dostępny jest dopiero w asynchronicznym wywołaniu zwrotnym (ang. callback), które jest całkowicie niedeterministyczne czasowo, to znaczy nie jest określona kolejność rozpoczęcia wczytywania obrazów. Protokół komunikacji i specyfika elemntu \emph{canvas} -- pozwalająca na rysowanie tylko na wierzchu elementu -- wymusza rysowanie elementów na widgetach w odpowiedniej kolejności. Jest to sprzeczne z asynchronicznością wywołań zwrotnych, dlatego przed rozpoczęciem rysowania danego widgetu ładowane są wszystkie potrzebne obrazki. Używana jest do tego metoda \emph{delayCallback} w listingu \ref{lst:source_images_delay}. Pierwszym argumentem jest liczba obrazków do załadowania, drugim funkcja zwrotna, która zostanie wywołana po załadowaniu się wszystkich obrazów. Funkcja zwraca domknięcie, które należy wywołać w każdym wywołaniu zwrotnym załadowania obrazu (\emph{onload}).

\label{lst:source_images_delay}
\begin{lstlisting}[language=JavaScript,numbers=left,caption={Metoda łącząca wywołania wielu callbacków w jeden},label={lst:source_images_delay}]
var delayCallback = function (num, callback) {
    function finished() {
        num -= 1;
        if (num == 0 && callback) {
            callback();
        }
    }
    return finished;
};
\end{lstlisting}

