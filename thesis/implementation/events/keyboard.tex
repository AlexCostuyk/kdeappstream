W zdarzeniach klawiatury używane są następujące pola:
\begin{itemize}
\item type -- typ zdarzenia,
\item key -- kod klawisza
\item text -- ciąg znaków będący rezultatem zdarzenia
\item autorep -- wartość logiczna określająca, czy zdarzenie jest wynikiem przytrzymania klawisza
\item count -- ilość powtórzeń
\item modifiers -- analogicznie jak w przypadku zdarzeń myszy
\end{itemize}

\subsubsection{Wciśnięcie klawisza}
\begin{lstlisting}[language=JavaScript,numbers=none]
{
  "command": "key",
  "type": "press", 
  "key": 0x193,    // Kod klawisza
  "text": "a",     // Ciag znakow UTF8 bedacy rezultatem zdarzenia
  "autorep": true|false, // Okresla czy zdarzenie jest wynikiem
                         // przytrzymania klawisza przed dluzszy czas
  "count": 0,    // Okresla ile powtorzen klawisza mialo miejsce
  "modifiers": 0x0  // 0x00000000 Qt::NoModifier
                    // 0x02000000 Qt::ShiftModifier
                    // 0x04000000 Qt::ControlModifier
                    // 0x08000000 Qt::AltModifier
                    // 0x10000000 Qt::MetaModifier
                    // 0x20000000 Qt::KeypadModifier
                    // 0x40000000 Qt::GroupSwitchModifier
}
\end{lstlisting}

\subsubsection{Zwolnienie klawisza}
\begin{lstlisting}[language=JavaScript,numbers=none]
{
  "command": "key",
  "type": "release", 
  "key": 0x193,    // Kod klawisza
  "text": "a",     // Ciag znakow UTF8 bedacy rezultatem zdarzenia
  "autorep": true|false, // Okresla czy zdarzenie jest wynikiem
                         // przytrzymania klawisza przed dluzszy czas
  "count": 0,    // Okresla ile powtorzen klawisza mialo miejsce
  "modifiers": 0x0  // Qt::NoModifier
}
\end{lstlisting}
 