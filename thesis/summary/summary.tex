Temat pracy inżynierskiej został w pełni zrealizowany, a jej wynikiem jest prototyp serwera oraz webowej aplikacji klienckiej. 

Program serwera udostępnia kod strony WWW, który uruchamiany jest przez przeglądarkę. Uruchamia on również aplikację graficzną opartą o framework Qt oraz odpowiada za utworzenie połączenia między klientem a procesem aplikacji. Możliwa jest także obsługa wielu połączeń równocześnie. 
Przy pomocy techniki wstrzykiwania kodu bibliotek linkowanych dynamicznie \emph{(ang. DLL injection)} oraz wewnętrznych mechanizmów biblioteki \emph{Qt}, użycie aplikacji nie wymaga ponownej kompilacji, zarówno samego frameworka \emph{Qt}, jak i uruchamianych aplikacji użytkowych.

Aplikacja kliencka stworzona w postaci dynamicznej strony WWW pokrywa bardzo duży podzbiór funkcjonalności modułu graficznych interfejsów biblioteki \emph{Qt}. Wszelkie braki wynikają z niedoskonałości standardu HTML5, który wciąż jest mocno rozwijany.

Projekt będzie kontynuowany w następujących kierunkach:
\begin{itemize}
  \item umożliwienie współpracy z aplikacjami opartymi o najnowszą bibliotekę \emph{Qt} w wersji \emph{5.0},
  \item rozwinięcie zabezpieczeń --- autentykacja i autoryzacja klientów,
  \item stworzenie panelu administracyjnego serwera oraz nowego widoku głównego aplikacji,
  \item automatyzacja procesu instalacji,
  \item utworzenie wersji serwera dla systemów \emph{Windows} oraz \emph{Mac OS},
  \item rozwinięcie możliwości aplikacji klienckiej przy wykorzystaniu elementów technologii \emph{HTML5}, które będą dostępne w przyszłości.
\end{itemize}

Stworzona implementacja jest w pełni otwarta --- kod źródłowy dostępny jest publicznie pod adresem \url{https://gitorious.org/kde-appstream/kde-appstream}.