
Każdy element graficznego interfejsu aplikacji (QWidget) jest renderowany w momencie odebrania zdarzenia QPaintEvent z kolejki zdarzeń głównego wątku aplikacji. Dzięki temu istnieje łatwy sposób na uzyskanie informacji o tym kiedy oraz ktory element należy przerenderować aby uaktualnić jego wygląd po stronie klienta. Problemem w dalszyb ciągu pozostaje jednak sposób na uzyskanie informacji o samym wyglądzie. 

Proponowane rozwiązanie polega na zaimplementowaniu abstrakcyjnego urządzenia wyjściowego reprezentującego przeglądarkę WWW po stronie klienta (patrz podrozdział \ref{system_rysowania}). Odpowiednio implementując klasy \emph{QPaintEngine} oraz \emph{QPaintDevice} możliwe staje się uzyskanie szczegółowych informacji dotyczących wygądu widgetów co z kolei umozliwia stworzenie innowacyjnego formatu przesyłanych danych. Zamiast przesyłać bitmapy z wyrenderowanym elementem można wysłać informację o kolorach, punktach, liniach i innych podstawowych elementach, które zostaną narysowane na urządzeniu docelowym jakim po stronie klienta jest przeglądarka WWW z obsługą elementów \emph{canvas}.


// TODO Komentarz Co to znaczy innowacyjnego?