\section{Komunikacja między klientem a serwerem}
%Specyfiką problemu jest jego dwuetapowość. W pierwszym etapipe klient inicjuje połączenie jednorazowo wysyłając zapytanie zawierające informacje o aplikacji, którą klient chce uruchomić oraz identyfikatorze klienta. W drugiej kolejności wymagane jest utworzenie kanału komunikacyjnego między klientem a procesem aplikacji. 

Do realizacji tego zadania stworzony został prosty serwer działający w oparciu o protokół HTTP. Jako zasób domyślny udostępnia on listę dostępnych aplikacji, które klient może uruchomić. Lista ta jest w pełni konfigurowalna po stronie serwera. Inicjalizacja połączenia polega na wysłaniu przez klienta nazwy wybranej aplikacji. Serwer po pomyślnej weryfikacji przydziela klientowi unikatowy identyfikator sesji, uruchamia proces aplikacji i wysyła klientowi skrypt w języku JavaScript zajmujący się przetwarzaniem po stronie klienta.

\section{Komunikacja między klientem a aplikacją}

// TODO: Opis do cz. teoretycznej (pogrubione)

Do rozwiązania tego problemu konieczne jest utworzenie ciągłego kanału komunikacyjnego między klientem a procesem aplikacji, za pomocą którego będzie możliwe przesyłanie informacji o wyglądzie interfejsu aplikacji oraz informowanie aplikacji o zdarzeniach generowanych przez użytkownika po stronie przeglądarki. Jako, że za cel przyjęte zostało założenie o nieingerowaniu bezpośrednio w kod skompilowanych już aplikacji, postawiono \bf{na technikę umożliwiającą} załadowanie kodu biblioteki dynamicznej do przestrzeni pamięciowej procesu aplikacji tuż przed jego uruchomieniem. Kod ten ma za zadanie utrzymanie połączenia oraz transmisję danych między klientem a aplikacją.