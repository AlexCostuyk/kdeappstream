Interakcja użytkownika z aplikacją sprowadza się do obsługi następujących zdarzeń:
\begin{enumerate}
  \item ruch myszy nad elementem,
  \item wciśnięcie, zwolnienie oraz dwuklik przycisku myszy,
  \item zmiana położenia kółka myszy,
  \item wciśnięcie oraz zwolnienie klawiszy na klawiaturze,
  \item zmiana rozmiaru okna aplikacji poprzez przeciąganie jego krawędzi,
  \item zamknięcie, minimalizacja lub maksymalizacja okna aplikacji.
\end{enumerate}
Większość z wyżej wymienionych elementów jest obsługiwana jako zdarzenia w języku JavaScript większości dzisiejszych przeglądarek. Proponowane podejście na rozwiązanie tego zagadnienia polega na stworzeniu formatu danych bazując na notacji JSON (JavaScript Object Notation). Dane w tym formacie przesyłane do serwera są następnie poddawane walidacji i konwersji na obiekty zdarzeń biblioteki Qt. Zdarzenia takie są następnie przesyłane do kolejki zdarzeń w głównym wątku aplikacji.

Odbiorcą zdarzenia jest widget, który wygenerował dane zdarzenie po stronie przeglądarki bazując na hierarchicznej budowie interfejsu użytkownika. Wyjątkami są tutaj zdarzenia klawiatury, które nie mają bezpośredniego odbiorcy w momencie ich zaistnienia. Aplikacja sama decyduje o tym, który element powinien odebrać zdarzenie. Domyślnie jest to widget, który atualnie posiada tzw. focus, a to z kolei zależy od poprzednich zdarzeń oraz logiki samego programu. W celu symulacji podobnego zachowania decyzja o odbiorcy zdarzeń klawiatury podejmowana jest po stronie serwera bazując na aktualnym stanie aplikacji.

// TODO: Przepływy danych / d. kolaboracji (nie wiem co to drugie ma znaczyć :P)